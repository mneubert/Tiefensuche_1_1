\section{"`Hilfss�tze"' - Spezifikation und Module}
\label{sec:digraph-path}

Bei der Bearbeitung des Projektes und der Beweisf�hrung mussten verschiedene Ans�tze aus der Graphenbibliothek zusammengef�hrt sowie Teilbeweise ausgelagert werden, da sie in mehreren Beweisen notwendig waren.\\
Zun�chst werde die Spezifikation "`union\_bool\_graph"' betrachtet, welche die notwendigen Pakete der Graphenbibliothek zusammenfasst. Folgenden die verwendete Spezifikation:

union\_bool\_graph = digraph-outsorted-neighbours + bool

Eine weitere Spezifikation ist "`digraph-path"', in welcher die in Kapitel 3.1 erw�hnte Signatur "`getM"' definiert wird.

digraph-path = \\
{\bf enr}\={\bf ich} union\_bool\_graph {\bf with}\+\\
{\bf functions} getM  : elem $\times$ digraph \Imp  nat ;
{\bf axioms}

\begin{quote}
getM(a, @g) = 0;

getM(a, @g +v a) = 1;

\Not b $\in$\_v g \And a $\in$\_v g \Imp getM(a, g +e a --$>$ b) = getM(a, g) + 1;

\end{quote}
{\bf end enrich}

Die Methode "`getM"' bekommt sowohl einen Knoten a sowie einen Graphen g �bergeben und gibt eine Zahl m zur�ck. Diese bezeichnet hierbei die Anzahl der von dem Knoten a im Graphen g erreichbaren Knoten, wobei sowohl direkt erreichbare als auch �ber transitive Kanten erreichbare ber�cksichtigt werden. Auch hier werden wieder drei Axiome definiert:

\begin{itemize}
	\item Axiom 1 gibt dann, dass in einem leeren Graphen kein Knoten von a erreicht werden kann.
	\item In Axiom 2 wird der leere Graph um einen Knoten a erweitert. Die Anzahl der von a erreichbaren Knoten ist in diesem Fall 1.
	\item Im dritten Axiom wird ein Knoten b, welcher noch nicht im Graphen lag sowie eine Kante vom Knoten a zu Knoten b hinzugef�gt. Die Anzahl der erreichbaren Knoten entspricht nun der Anzahl der erreichbaren Knoten ohne den Knoten b plus 1. 
\end{itemize}
Die Spezifikation wird durch das Modul "`dfs-temp"' implementiert. Folgend ist die Implementierung der Funktion "`getM"' zu sehen.

\medskip

getM\#(a, g; {\bf var} n) \\
\{\\
{\bf let} \Var{s}{20} = $\emptyset$, \Var{b}{99} = a {\bf in}\\
{\bf \{}\\
\tabbe n := 0;\\
\tabbe \Var{s}{20} := g.vs;\\
\tabbe {\bf while} \Var{s}{20} $\neq$ $\emptyset$ {\bf do}\\
\tabbe \tab{wh} \{\Var{b}{99} := \Var{s}{20}.max; \Var{s}{20} := \Var{s}{20}.butmax; {\bf if} path$\exists$(a, \Var{b}{99}, g) {\bf then} n := n + 1\}\\
{\bf \}}\\
\}

F�r einen gegebenen Knoten werden in einer Schleife alle anderen Knoten des Graphen betrachtet. Mit der Funktion "`path$\exists$"' aus der Graphenbibliothek wird dann �berpr�ft, ob es einen Pfad zwischen den beiden Knoten gibt. Wenn ja, wird die Z�hlvariable erh�ht. \\
Die Beweise der Axiome gelangen alle, wie sich in folgendem Abschnitt sehen l�sst. 
\medskip

Term-getmjsiz: 
 \Fol \Do getM\#\Dc\ true


uses: (decl)\\
used by: (Exp-01 Exp-03)\\
Proof steps: 50, interactions: 50

\medskip

Exp-01: 
\begin{flushleft}


\Fol

\Not b $\in$\_v g \And a $\in$\_v g \Imp \Do getM\#\Dc \Do getM\#\Dc m = \Var{m}{0} + 1

\end{flushleft}


uses: (Term-getmjsiz)\\
Proof steps: 2, interactions: 1

\medskip

Exp-02: 
 \Fol \Do getM\#\Dc\ m = 0


uses: (decl)\\
Proof steps: 6, interactions: 6

\medskip

Exp-03: 
 \Fol \Do getM\#\Dc\ m = 1


uses: (Term-getmjsiz)\\
Proof steps: 1, interactions: 1

\medskip

Auch hier sei wieder auf die verwendeten Beweisschritte hingewiesen, welche in den folgenden Tabellen aufgef�hrt sind:

Statistic for each theorem:

\begin{supertabular}{l|r|r|r|r}
	& proof steps & interactions & automation & induction\\ \hline
Term-getmjsiz & 50 & 50 & 0.0 \% & nat\\
Exp-02 & 6 & 6 & 0.0 \% & ---\\
Exp-01 & 2 & 1 & 50.0 \% & ---\\
Exp-03 & 1 & 1 & 0.0 \% & ---\\

\end{supertabular}

Heuristic applications:

\begin{supertabular}{l|r|r}
heuristic	& total applications & percentage \\ \hline
Interactive & 57 & 98.2 \% \\
System & 1 & 1.7 \% \\

\end{supertabular}

Rule applications:

\begin{supertabular}{l|r|r}
rule	        & total applications & percentage \\ \hline
insert rewrite lemma & 14 & 24.1 \% \\
cut formula & 9 & 15.5 \% \\
assign right & 8 & 13.7 \% \\
simplifier & 6 & 10.3 \% \\
weakening & 3 & 5.1 \% \\
call right & 2 & 3.4 \% \\
split right & 2 & 3.4 \% \\
vardecls right & 2 & 3.4 \% \\
case distinction & 2 & 3.4 \% \\
if right & 1 & 1.7 \% \\
while exit right & 1 & 1.7 \% \\
all right & 1 & 1.7 \% \\
normalize & 1 & 1.7 \% \\
invariant right & 1 & 1.7 \% \\
insert lemma & 1 & 1.7 \% \\
skip right & 1 & 1.7 \% \\
exists left & 1 & 1.7 \% \\
induction & 1 & 1.7 \% \\
all left & 1 & 1.7 \% \\

\end{supertabular}

Lemma applications:

\begin{supertabular}{l|r|r}
theorem	        & total applications & percentage \\ \hline
decl & 2 & 66.6 \% \\
Term-getmjsiz & 1 & 33.3 \% \\

\end{supertabular}
