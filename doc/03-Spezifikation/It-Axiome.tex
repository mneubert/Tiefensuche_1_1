\subsection{Spezifikation \texttt{digraph-dfs-proc}}
\label{sec:digraph-dfs-proc}

Ausgehend von den �berlegungen in Kapitel 2 soll zun�chst eine Tiefensuche als Prozedur umgesetzt werden.
Dies bedeutet, dass die R�ckgabe des Programmes ein Wahrheitswert ist. Enth�lt der Graph g einen Pfad vom Startknoten zur Zielknotenmenge, so
wird \texttt{true} zur�ckgeliefert, anderfalls \texttt{false}. Im folgenden Absatz sieht man die Signatur der Spezifikation:

digraph-dfs-proc = \\
{\bf enrich} digraph-basic, union\_bool\_graph {\bf with}\+\\
{\bf predicates} deptSearchP  : elem $\times$ list $\times$ digraph;

{\bf axioms}

$\vdash$ z = [] \And a $\in$\_v g \Or a ' = [] \Imp (deptSearchP(a, z, g) \Equiv false);

$\vdash$ pathp(x, g) \And x.first = a \And x.last = b \Imp (deptSearchP(a, b ', g) \Equiv true);

$\vdash$ a --$>$ b $\in$\_e g \And b $\in$ z \Imp (deptSearchP(a, z, g) \Equiv true);

$\vdash$ a --$>$ b $\in$\_e g \And b --$>$ c $\in$\_e g \And c $\in$ z \Imp (deptSearchP(a, z, g) \Equiv true);

{\bf end enrich}

Die Spezifikation erweitert hierbei die vorhandenen Spezifikationen \texttt{digraph-basic} sowie \texttt{union\_bool\_graph}, welche Hilfss�tze enthalten und im Abschnitt \ref{sec:digraph-path} erl�utert werden.
Die in der Spezifikation definierte Prozedur \texttt{deptSearchP} bildet dabei drei Eingabeparameter \texttt{elem} $\times$ \texttt{list} $\times$ \texttt{digraph} auf einen Wahrheitswert ab.
Dabei bezeichnet \texttt{elem} den Startknoten, \texttt{list} die Menge der Zielknoten und \texttt{digraph} den Graphen.
Des Weiteren werden in der Spezifikation verschiedene Axiome deklariert:\\
\begin{itemize}
\item Das erste Axiom beschreibt den Fall, dass unzul�ssige Parameter beim Prozeduraufruf �bergeben werden. Diese Situation tritt auf, wenn keine Zielknoten angegeben werden oder a ein leerer Knoten ist. In dieser Konstelation existiert kein Pfad im Graph und somit soll die Prozedur \texttt{false} zur�ckgeben.  
\item Das zweite Axiom nimmt Bezug auf die Signatur \texttt{pathp}, welche in der Graphenbibliothek implementiert wurde und �berpr�ft, ob es einen Pfad x im Graphen g gibt. Ist dies der Fall und der Anfangsknoten des Pfades stimmt mit dem Startknoten a �berein und der letzte Knoten des Pfades ist ein Element der Zielknotenmenge, dann liefert die Prozedur \texttt{deptSearchP} \texttt{true} zur�ck.
\item Im dritten Axiom wird der Bezug zum Begriff der Kante im Graphen gebildet. Wenn es eine Kante von Knoten a nach Knoten b im Graphen g gibt und b in der Zielknotenmenge liegt, dann muss die Methode \texttt{deptSearchP} diesen Weg finden. Dieses Axiom ist ein Spezialfall des vierten Axioms.
\item Im vierten Axiom wird der Begriff der Kante um die transitive Eigenschaft erweitert. Die Prozedur muss also auch dann einen Pfad finden bzw. \texttt{true} zur�ckliefern, wenn es von Knoten a �ber einen anderen Knoten b einen Weg in die Zielknotenmenge gibt. Setzt man in diesem Fall b = c, hat man den Spezialfall, welcher in Axiom 3 beschrieben wurde. 
\end{itemize}




