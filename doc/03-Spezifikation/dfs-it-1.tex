\section{Modul "`dfs-it-1"'}

Das Modul "`dfs-it-1"' implementiert die Spezifikation "`digraph-dfs-proc"', also die Tiefensuche als Prozedur mit der R�ckgabe eines Wahrheitswertes. Die genaue Implementierung ist unten aufgezeigt. Allgemein kann gesagt werden, dass zun�chst die fehlerhaften Eingaben abgefangen werden, bevor der eigentliche Algorithmus zum Einsatz kommt. Dabei wird in einer Schleife so lange ein neuer Knoten expandiert, bis entweder ein Pfad gefunden wurde, oder alle erreichbaren Knoten abgearbeitet wurden. F�r letzteren Fall wurde die Hilfsfunktion "`getM"' einegf�hrt, welche im Kapitel 3.xx noch genauer erl�utert wird. In der while-Schleife wird wie erw�hnt ein Knoten von der open-Liste untersucht und geschaut, ob von ihm erreichbare Knoten noch nicht auf der close-Liste sind. Dazu wird die Funktion "`outsortSuccs"' verwendet, welche aus der Graphenbibliothek stammt.\\
Unterhalb der Deklaration findet man noch einmal die Axiome. Hierbei wird aufgelistet, in wie weit die Axiome mit der Implementierung bewiesen werden konnten und wie die Abh�ngigkeiten untereinander sind. 

{\parindent0cm


{\LARGE\bf decl}\label{lemma-decl}

\medskip

deptSearchP\#(a, z, g; {\bf var} bbool) \\
\{\\
{\bf let} open = [] + a, close = [], \Var{n}{0} = 0, \Var{a}{0} = a {\bf in}\\
{\bf \{}\\
\tabbe bbool := false;\\
\tabbe \Var{n}{0} := getM(a, g) + 1;\\
\tabbe {\bf if} \Not (a ' = [] \Or z = []) {\bf then}\\
\tabbe \tabif\  {\bf if} a $\in$ z {\bf then} bbool := true {\bf else}\\
\tabbe \tabif\  {\bf \{}\\
\tabbe \tabif\ \tabbe {\bf while} \Var{n}{0} $>$ 0 \And (bbool \Equiv false) \And (open $\neq$ []) {\bf do}\\
\tabbe \tabif\ \tabbe \tab{wh} {\bf \{}\\
\tabbe \tabif\ \tabbe \tab{wh}\tabbe \Var{a}{0} := open.first;\\
\tabbe \tabif\ \tabbe \tab{wh}\tabbe {\bf if} \Var{a}{0} $\in$ z {\bf then} bbool := true {\bf else}\\
\tabbe \tabif\ \tabbe \tab{wh}\tabbe  {\bf \{}\\
\tabbe \tabif\ \tabbe \tab{wh}\tabbe \tabbe \Var{n}{0} := \Var{n}{0} - 1;\\
\tabbe \tabif\ \tabbe \tab{wh}\tabbe \tabbe close := \Var{a}{0} ' + close;\\
\tabbe \tabif\ \tabbe \tab{wh}\tabbe \tabbe open := set2list(outsortSuccs(\Var{a}{0}, list2set(close), g)) + open.rest\\
\tabbe \tabif\ \tabbe \tab{wh}\tabbe {\bf \}}\\
\tabbe \tabif\ \tabbe \tab{wh}{\bf \}};\\
\tabbe \tabif\ \tabbe {\bf if} \Var{n}{0} = 0 {\bf then} bbool := false\\
\tabbe \tabif\ {\bf \}}\\
{\bf \}}\\
\}

\medskip

{\LARGE\bf Term-deptsearchpjsiz}\label{lemma-Term-deptsearchpjsiz}

\medskip

 \Fol \Do deptSearchP\#\Dc true

\begin{itemize}

\item Type: a proof obligation

\item       must be proved
\item       a complete proof exists
\item       used lemmas  : decl
\item       used by      : Exp-02, Exp-03, Exp-04
\item       interactions : 47
\item       proof steps  : 47
\end{itemize}

\medskip

{\LARGE\bf Exp-01}\label{lemma-Exp-01}

\medskip

 \Fol z = [] \And a $\in$\_v g \Or a ' = [] \Imp (\Do deptSearchP\#\Dc bbool \Equiv false)

\begin{itemize}

\item Type: a proof obligation

\item       must be proved
\item       a complete proof exists
\item       used lemmas  : decl
\item       interactions : 0
\item       proof steps  : 8
\end{itemize}

\medskip

{\LARGE\bf Exp-02}\label{lemma-Exp-02}

\medskip

 \Fol pathp(x, g) \And x.first = a \And x.last = b \Imp (\Do deptSearchP\#\Dc bbool \Equiv true)

\begin{itemize}

\item Type: a proof obligation

\item       must be proved
\item       a complete proof exists
\item       used lemmas  : Term-deptsearchpjsiz
\item       interactions : 1
\item       proof steps  : 2
\end{itemize}

\medskip

{\LARGE\bf Exp-03}\label{lemma-Exp-03}

\medskip

 \Fol a --$>$ b $\in$\_e g \And b $\in$ z \Imp (\Do deptSearchP\#\Dc bbool \Equiv true)

\begin{itemize}

\item Type: a proof obligation

\item       must be proved
\item       a complete proof exists
\item       used lemmas  : Term-deptsearchpjsiz
\item       interactions : 3
\item       proof steps  : 4
\end{itemize}

\medskip

{\LARGE\bf Exp-04}\label{lemma-Exp-04}

\medskip

 \Fol a --$>$ b $\in$\_e g \And b --$>$ c $\in$\_e g \And c $\in$ z \Imp (\Do deptSearchP\#\Dc bbool \Equiv true)

\begin{itemize}

\item Type: a proof obligation

\item       must be proved
\item       a complete proof exists
\item       used lemmas  : Term-deptsearchpjsiz
\item       interactions : 1
\item       proof steps  : 2
\end{itemize}

\medskip

Wie zu sehen ist, konnten alle Axiome bewiesen werden. Dabei wurde bei den Axiomen 2 bis 4 auf den Terminationsbeweis zur�ck gegriffen. In folgendem Abschnitt sind noch einmal die bei den Beweisen verwendeten Beweisschritte aufgef�hrt.

Statistic for each theorem:

\begin{supertabular}{l|r|r|r|r}
	& proof steps & interactions & automation & induction\\ \hline
Term-deptsearchpjsiz & 47 & 47 & 0.0 \% & nat\\
Exp-01 & 8 & 0 & 100.0 \% & ---\\
Exp-03 & 4 & 3 & 25.0 \% & ---\\
Exp-04 & 2 & 1 & 50.0 \% & ---\\
Exp-02 & 2 & 1 & 50.0 \% & ---\\

\end{supertabular}

Heuristic applications:

\begin{supertabular}{l|r|r}
heuristic	& total applications & percentage \\ \hline
Interactive & 52 & 82.5 \% \\
symbolic execution & 5 & 7.9 \% \\
System & 4 & 6.3 \% \\
simplifier & 1 & 1.5 \% \\
weak unfold & 1 & 1.5 \% \\

\end{supertabular}

Rule applications:

\begin{supertabular}{l|r|r}
rule	        & total applications & percentage \\ \hline
assign right & 13 & 20.6 \% \\
simplifier & 8 & 12.6 \% \\
if right & 6 & 9.5 \% \\
normalize & 4 & 6.3 \% \\
insert lemma & 4 & 6.3 \% \\
skip right & 4 & 6.3 \% \\
assign left & 3 & 4.7 \% \\
while exit right & 3 & 4.7 \% \\
insert elim lemma & 2 & 3.1 \% \\
split left & 2 & 3.1 \% \\
while right & 1 & 1.5 \% \\
case distinction & 1 & 1.5 \% \\
cut formula & 1 & 1.5 \% \\
skip left & 1 & 1.5 \% \\
apply induction & 1 & 1.5 \% \\
if negative left & 1 & 1.5 \% \\
induction & 1 & 1.5 \% \\
call left & 1 & 1.5 \% \\
weakening & 1 & 1.5 \% \\
vardecls left & 1 & 1.5 \% \\
split right & 1 & 1.5 \% \\
call right & 1 & 1.5 \% \\
vardecls right & 1 & 1.5 \% \\
execute while & 1 & 1.5 \% \\

\end{supertabular}

Lemma applications:

\begin{supertabular}{l|r|r}
theorem	        & total applications & percentage \\ \hline
Term-deptsearchpjsiz & 4 & 66.6 \% \\
decl & 2 & 33.3 \% \\

\end{supertabular}