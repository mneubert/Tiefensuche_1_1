\subsection{Spezifikation "`digraph-dfs-func"'}

Im Gegensatz zu den vorangegangenen Kapiteln, bei dem die Tiefensuche als
Prozedur spezifiziert und implementiert wurde, wird nun der Algorithmus als
Funktion umgesetzt. Das hei�t, dass sowohl die iterative als auch die rekursive
Variante einen R�ckgabeparameter besitzen, der einen Pfad vom Startknoten a
in die Zielknotenmenge z enth�lt, wenn dieser im Graphen G existiert.
Andernfalls wird die leere Liste zur�ckgegeben.
\par
Der folgende Absatz zeigt die Signatur der Spezifikation:

\begin{tabbing}\label{digraph-dfs-func-spec}%
digraph-dfs-func = \\
{\bf enr}\={\bf ich} union\_bool\_graph {\bf with}\+\\
{\bf functions} deptSearchF  : elem $\times$ list $\times$ digraph \Imp  list ;
\end{tabbing}
{\bf axioms}



\begin{quote}
z = [] \And a $\in$\_v g \Or a ' = [] \Imp deptSearchF(a, z, g) = [];

pathp(x, g) \And x.first = a \And x.last = b \Imp deptSearchF(a, b ', g) = x;

a --$>$ b $\in$\_e g \And b $\in$ z \Imp deptSearchF(a, z, g).first = a;

a --$>$ b $\in$\_e g \And b $\in$ z \Imp deptSearchF(a, z, g).last = b;

a --$>$ b $\in$\_e g \And b --$>$ c $\in$\_e g \And c $\in$ z \Imp deptSearchF(a, z, g).first = a;

a --$>$ b $\in$\_e g \And b --$>$ c $\in$\_e g \And c $\in$ z \Imp deptSearchF(a, z, g).last = c;


\end{quote}
{\bf end enrich}

Die Spezifikation erweitert die Spezifikation \emph{union\_bool\_graph}, welche
in Kapitel \ref{} erl�utert wird. Innerhalb von \emph{digraph-dfs-func} wird die
Funktion \emph{deptSearchF} beschrieben, welche eine Abbildung von \emph{elem
$\times$ list $\times$ digraph $\rightarrow$ list} ist. Die definierten Axiome
werden im Folgenden entsprechend der obigen Ordnung erkl�rt:

\begin{itemize}
  \item Das erste Axiom besagt, dass wenn der Funktion entweder keine
  Zielknotenmenge oder kein Startelement �bergeben wird, diese einen leeren Pfad
  zur�ckgibt, sprich die leere Liste.
  \item Das zweite Axiom stellt den Fall dar, dass es im Graphen eine Kante
  gibt, deren Startknoten a und deren Endknoten b ist. Wird die Funktion in der
  Form aufgerufen, dass a der Startknoten und b der Zielknoten ist, so ist der
  R�ckgabepfad die Kante x.
  \item Axiom drei und vier erf�llen dieselben Voraussetzungen wie zwei, mit dem
  Unterschied, dass die Aussage von Axiom 3 ist, dass der Start des
  R�ckgabepfades a ist und in Axiom vier das Ende Pfades Knoten b.
  \item Axiom f�nf bildet die Transitivit�t ab, wobei gesagt wird, dass wenn
  eine Kante von a nach b und von b nach c existiert, wobei c Zielknoten ist, der
  Knoten a das erste Element auf dem R�ckgabepfad ist.
  \item Analog zu Axiom f�nf, ist die Aussage das auf dem transitiven Pfad das
  letzte Element der Zielknoten c ist.
\end{itemize}