\subsection{Spezifikation \texttt{digraph-dfs-func}}
\label{sec:digraph-dfs-func}

Im Gegensatz zu den vorangegangenen Abschnitten \ref{sec:digraph-dfs-proc} bis
\ref{sec:dfs-rek-1}, bei denen die Tiefensuche als Prozedur mit
Wahrheitswertr�ckgabe spezifiziert und implementiert wurde, wird nun der
Algorithmus als Funktion umgesetzt. Das hei�t, dass sowohl die iterative als
auch die rekursive Variante einen R�ckgabeparameter besitzen, der einen Pfad
vom Startknoten a in die Zielknotenmenge z enth�lt, wenn dieser im Graphen g
existiert. Andernfalls wird die leere Liste zur�ckgegeben.
\par
Der folgende Absatz zeigt die Signatur der Spezifikation:

\begin{tabbing}
digraph-dfs-func = \\
{\bf enr}\={\bf ich} union\_bool\_graph {\bf with}\+\\
{\bf functions} deptSearchF  : elem $\times$ list $\times$ digraph \Imp  list ;
\end{tabbing}
{\bf axioms}



\begin{quote}
z = [] \And a $\in$\_v g \Or a ' = [] \Imp deptSearchF(a, z, g) = [];

pathp(x, g) \And x.first = a \And x.last = b \Imp deptSearchF(a, b ', g) = x;

a --$>$ b $\in$\_e g \And b $\in$ z \Imp deptSearchF(a, z, g).first = a;

a --$>$ b $\in$\_e g \And b $\in$ z \Imp deptSearchF(a, z, g).last = b;

a --$>$ b $\in$\_e g \And b --$>$ c $\in$\_e g \And c $\in$ z \Imp deptSearchF(a, z, g).first = a;

a --$>$ b $\in$\_e g \And b --$>$ c $\in$\_e g \And c $\in$ z \Imp deptSearchF(a, z, g).last = c;


\end{quote}
{\bf end enrich}

Die Spezifikation erweitert die Spezifikation \texttt{union\_bool\_graph},
welche in Abschnitt \ref{sec:digraph-path} erl�utert wird. Innerhalb von
\texttt{digraph-dfs-func} wird die Funktion \texttt{deptSearchF} beschrieben,
welche eine Abbildung von \texttt{elem $\times$ list $\times$ digraph
$\rightarrow$ list} ist. Die definierten Axiome werden im Folgenden
entsprechend der obigen Ordnung erkl�rt:

\begin{itemize}
  \item Das erste Axiom besagt, dass die Funktion einen leeren Pfad, d.h. eine
  leere Liste, zur�ckgibt, wenn ihr entweder eine leere Zielknotenmenge oder
  kein Startelement �bergeben wurde.
  \item Das zweite Axiom stellt den Fall dar, dass es im Graphen einen Pfad
  gibt, dessen Startknoten a und dessen Endknoten b ist. Wird die Funktion in
  der Form aufgerufen, dass a der Startknoten und b der Zielknoten ist, so ist der
  R�ckgabepfad der Pfad x.
  \item Axiom 3 und 4 erf�llen beide die Voraussetzung, dass es eine Kante von
  a nach b im Graphen gibt, mit a als Startknoten und b als Zielknoten. F�r
  Axiom 3 gilt dann, dass der Start des R�ckgabepfades Knoten a ist und in Axiom
  4 das Ende Pfades Knoten b ist.
  \item Axiom f�nf bildet die Transitivit�t ab. Unter der Voraussetzung, dass c
  ein Zielknoten ist und ein Pfad von a nach b und von b nach c existiert, ist
  der Knoten a das erste Element auf dem R�ckgabepfad.
  \item Analog zu Axiom f�nf, ist die Aussage das auf dem transitiven Pfad das
  letzte Element der Zielknoten c ist.
\end{itemize}