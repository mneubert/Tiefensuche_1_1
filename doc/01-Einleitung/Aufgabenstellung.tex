\section{Aufgabenstellung}

Innerhalb der Lehrveranstaltung Programmverifikation, ist es die Aufgabe, f�r
eine selbstgew�hlte Problemstellung, ein Verifikationsprojekt mit Hilfe der
Software \emph{KIV} (Karlsruhe Interactive Verifier) zu erstellen. F�r die
vorliegende Arbeit wurde diesbez�glich das Problem der Tiefensuche auf Graphen
betrachtet. Dabei handelt es sich um einen Algorithmus, der mittels der
Strategie der \glqq{}Tiefen-Zuerst-Suche\grqq{}, die Existenz von Pfaden - von
einem gegebenen Startknoten zu einer Zielknotenmenge - pr�ft respektive in
einer modifizierten Variante den dabei resultierenden Pfad zur�ckgibt. 
\par
Eine genaue Beschreibung des Algorithmus sowie der �berblick �ber die konkrete
Umsetzung des Projektes, wird Gegenstand der folgenden Kapitel sein.
