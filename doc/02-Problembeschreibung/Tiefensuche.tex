\section{Tiefensuche}
\label{sec:tiefensuche}

Die Tiefensuche ist ein Verfahren zum Traversieren von Graphen. Das Ziel besteht darin einen Knoten zu finden, der zu einer gegebenen Zielknotenmenge geh�rt. Die Suche ist genau dann erfolgreich, wenn ein Pfad von einem gegebenen Startknoten zu einem Zielknoten aus der Zielknotenmenge existiert.
\par
Es werden zun�chst vom Startknoten alle direkt erreichbaren Knoten ermittelt, d.h. Knoten, die durch eine Kante mit dem Startknoten verbunden sind. Diese Knoten werden in eine open-Liste eingetragen. Auf der open-Liste befinden sich alle Knoten, die von einem expandierten Knoten erreichbar sind, selbst aber noch nicht expandiert wurden. Unter "`expandieren"' ist das Ermitteln der Nachbarknoten zu verstehen. Knoten, die bereits expandiert wurden, werden in die close-Liste eingetragen, um sie nicht nochmals zu untersuchen. Bei der Expansion eines Knotens werden nur Knoten ber�cksichtigt, die sich noch nicht auf der close-Liste befinden. Alle entdeckten Nachfolgerknoten werden vorn an die open-Liste angehangen. Daher kann die open-Liste als Stack betrachtet werden (LIFO-Prinzip). Durch die Verwendung dieser Datenstruktur unterscheidet sich die Tiefen- von der Breitensuche.
\par
Von der open-Liste wird nun immer der erste Knoten entnommen, auf die close-Liste gesetzt und expandiert. Die erhaltenen Nachfolgerknoten werden jeweils der open-Liste hinzugef�gt.
\par
Die Suche endet, sobald ein Knoten aus der Zielknotenmenge erreicht wurde oder die open-Liste leer ist. Im letztgenannten Fall existiert kein Pfad vom Startknoten zu einem Zielknoten.
\par
Die Funktionsweise wird anhand des in Abb. \ref{fig:dfs-ex} gezeigten Beispielgraphen verdeutlicht. Dabei ist a der Startknoten und e der einzige Knoten in der Zielknotenmenge. In Tabelle \ref{tab:dfs-ex} sind alle Bearbeitungsschritte der Tiefensuche aufgef�hrt. Es wird jeweils angegeben, welcher Knoten bearbeitet wird und welche Elemente sich nach der Bearbeitung dieses Knotens auf der open- und close-Liste befinden.

\begin{figure}[!ht]
\centering
\begin{tikzpicture}
	\node[vertex] (1) at (1,0) {a};
	\node[vertex] (2) at (4,1) {b};
	\node[vertex] (3) at (6,-1) {d};
	\node[vertex] (4) at (6,1) {c};
	\node[vertex] (5) at (8,0) {e};
	%\node (3) at (4,0) {...};
	\path [-] (1) edge node [above] {} (2);
	\path [-] (1) edge node [above] {} (3);
	\path [-] (2) edge node [above] {} (4);
	\path [-] (3) edge node [above] {} (4);
	\path [-] (3) edge node [above] {} (5);
	\path [-] (4) edge node [above] {} (5);	
\end{tikzpicture}
\caption{Beispielgraph}
\label{fig:dfs-ex}
\end{figure}

\begin{table}[!htp]
\begin{center}
\begin{tabular}{c|p{2cm}|p{2cm}}
  %\hline
  bearb. Knoten & open & close \\
  \hline
  Init. & [a] & [] \\
  a & [b, d] & [a] \\
  b & [c, d] & [a, b] \\
  c & [e, d] & [a, b, c] \\
  e & \multicolumn{2}{c}{Abbruch mit R�ckgabewert \texttt{true}} \\
	\end{tabular}
	\caption{Abarbeitungschritte f�r Beispielgraphen}%
	\label{tab:dfs-ex}%
	\end{center}
\end{table}