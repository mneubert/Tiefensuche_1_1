\section{Implementierungen}

\par
\vspace{0.2cm}
\textbf{Rekursive Implementierung}
\par

Die Tiefensuche kann rekursiv wie in Listing \ref{lst:dfs-rek} als Pseudocode dargestellt implementiert werden. Die Funktion wird mit den Paramentern \texttt{open = Startknoten}, \texttt{close = leere Liste}, \texttt{ziel = Zielknotenmenge} und \texttt{graph = zu untersuchender Graph} aufgerufen.
\par
Zu Beginn wird gepr�ft, ob sich der aktuell betrachtete Knoten in der Zielknoten befindet. Ist dies der Fall wird \texttt{true} als Ergebnis zur�ckgegeben. Ansonsten wird der Knoten der close-Liste hinzugef�gt und expandiert. Auf die Implementierung der Funktion \texttt{expand} wird nicht n�her eingegangen. Sie bestimmt die Nachbarn eines Knotens unter Ausschluss derjenigen, die sich bereits auf der close-Liste befinden.
\par
Ist die open-Liste nach diesem Bearbeitungsschritt nicht leer, so erfolgt ein rekursiver Aufruf der Funktion.

\begin{figure}[!h]
\begin{lstlisting}[caption=Pseudocode rekursive Tiefensuche, label=lst:dfs-rek]
dfs-rek(open, close, ziel, graph)
{
  knoten = open.pop;
  if (knoten in ziel)
  {
    return true;
  }
  else
  {
    close := knoten + close;
    open := expand (knoten, close) + open;
    
    if(open.isEmpty == false)
    {
      dfs-rek(open, close, ziel, graph);
    }
    else
    {
    	return false;
    }
  }
}

\end{lstlisting}
\end{figure}

\par
\vspace{0.2cm}
\textbf{Iterative Implementierung}
\par

Eine iterative Variante der Tiefensuche ist in Listing \ref{lst:dfs-it} als Pseudocode dargestellt. Die Aufrufparameter der Funktion sind der Startknoten (\texttt{startknoten}), die Zielknotenmenge (\texttt{ziel}) und der zu untersuchende Graph (\texttt{graph}). 
\par
Nach der Initialisierung der open- und close-Liste wird eine while-Schleife solange durchlaufen, bis die open-Liste leer ist oder vorzeitig mit der R�ckgabe von \texttt{true} abgebrochen wird. In der Schleife wird der erste Knoten der open-Liste entnommen und gepr�ft, ob dieser zur Zielknotenmenge geh�rt. Ist dies der Fall war die Suche erfolgreich und es wird abgebrochen. Andernfalls wird der Knoten der close-Liste hinzugef�gt und expandiert.
\par
Ist die open-Liste nach diesem Schritt leer, wird die Schleife verlassen und \texttt{false} zur�ckgegeben, da die Suche erfolglos war. Ansonsten erfolgt der n�chste Schleifendurchlauf.

\begin{figure}[!h]
\begin{lstlisting}[caption=Pseudocode iterative Tiefensuche, label=lst:dfs-it]
dfs-it(startknoten, ziel, graph)
{
  open := startknoten;
  close := [];
  while(open != empty)
  {
    knoten := open.pop;
    if(knoten in ziel)
    {
      return true;
    }
    else
    {
      close := knoten + close;
      open := expand(knoten, close) + open;
    }  
  }
  return false;
}
\end{lstlisting}
\end{figure}