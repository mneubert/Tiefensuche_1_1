\chapter{Zusammenfassung}
\label{chap:zusammenfassung}

Unser Ziel im Rahmen des Projektes eine sinnvolle Verwendung bzw. der Graphenbibliothek zu erreichen, konnte unserer Meinung nach voll umgesetzt werden. Mit der Implementierung und dem Beweis der Tiefensuche ist es uns gelungen, einen der g�ngigen Suchalgorithmen f�r Graphen zu verifizieren. Dabei ist es uns gelungen, verschiedene Varianten des Suchverfahrens umzusetzen. Die iterative Variante baut dabei vor allem auf die Abarbeitung von Schleifen auf, die rekursive Variante ben�tigt die Beweise von verschachtelten Funktionsaufrufen. Mit diesen verschiedenen Varianten ist es uns gelungen, einen gro�en Teil der M�glichkeiten von KIV zu erforschen. Dabei sei aber auch angemerkt, dass KIV ein sehr komplexes und m�chtiges Werkzeug ist. �ber das blo�e Grundverst�ndnis hinaus muss man sich f�r jeden Anwendungsfall spezifische L�sungsszenarien aneignen und diese geschickt kombinieren. Dies gelang und leider auch nicht in allen F�llen, so dass zumindest ein Beweis nicht abgeschlossen werden konnte. Leider machte uns an einigen Stelle auch KIV selbst die Arbeit schwer, da es in einigen Bereichen nicht so funktioniert, wie man es sich intuitiv erhofft. Alles in allem war das Projekt aber sehr f�rderlich, um unser Verst�ndnis f�r die Verifikation von Programmen zu erweitern.  