\chapter{Zusammenfassung}
\label{chap:zusammenfassung}

Das Ziel im Rahmen des Projektes eine sinnvolle Verwendung bzw. eine Anwendung der Graphenbibliothek zu erreichen, konnte unserer Meinung nach voll umgesetzt werden. Mit der Implementierung und dem Beweis der Tiefensuche ist es uns gelungen, einen der gängigen Suchalgorithmen für Graphen zu verifizieren. Dabei war es uns möglich, verschiedene Varianten des Suchverfahrens umzusetzen. Die iterative Variante baut dabei vor allem auf die Abarbeitung von Schleifen auf, die rekursive Variante benötigt die Beweise von verschachtelten Funktionsaufrufen. Mit diesen verschiedenen Varianten konnte ein großer Teil der Möglichkeiten von KIV erforscht werden. Dabei sei aber auch angemerkt, dass KIV ein sehr komplexes und mächtiges Werkzeug ist. Über das bloße Grundverständnis hinaus muss man sich für jeden Anwendungsfall spezifische Lösungsszenarien aneignen und diese geschickt kombinieren. Dies gelang leider auch nicht in allen Fällen, so dass zwei Beweise nicht abgeschlossen werden konnte. Leider machte uns an einigen Stelle auch KIV selbst die Arbeit schwer, da es in einigen Bereichen nicht so funktioniert, wie man es sich intuitiv erhofft. Alles in allem war das Projekt aber sehr förderlich, um unser Verständnis für die Verifikation von Programmen zu erweitern.  
